\documentclass{article}

\usepackage{tabu}
\begin{document}

\title{Szenario 1: Relativ milde Entwicklungen}

\maketitle





\subsection{Bewertung}\label{H348924}



\begin{tabu} to \textwidth { |X|X|X|X| }
\hline



Übertragung   & Immunevasion & Wesentliche Schwere & Erlebter Schwergrad
 \\


Orange & Grün & Orange & Grün
 \\
\hline

\end{tabu}




\subsection{Beschreibung}\label{H7334057}



Weitere SARS-CoV-II Varianten tauchen auf, aber es gibt keine größere antigene Evolution, keine Zunahme der Übertragbarkeit oder eine Rückkehr zum Delta-Niveau der intrinsischen Schwere. 


Minimales weitere Evasion des Schutzes der aktuellen Impfstoffen und infektionsinduzierter Immunität. 


Geringfügige saisonale/regionale Ausbrüche aufgrund nachlassender Immunität und geringfügiger antigener Veränderungen. 


Bestehende Impfstoffe werden jährlich nur zur Auffrischung von gefährdeten Personen eingesetzt. 


Virostatika haben einen erheblichen Einfluss auf die Mortalität und Morbidität und bleiben wirksam. 


In Jahren mit höheren SARS-CoV-2-Wellen treten tendenziell weniger Grippefälle auf. 


\textbf{Es wir erwartet für die nächsten 12-18 Monaten:} Relativ geringes Wiederaufflammen im Herbst/Winter 2022/23 mit nur geringer Anzahl an schweren Erkrankungen in der Bevölkerung.

\end{document}
