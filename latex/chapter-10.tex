\documentclass{article}

\usepackage{tabu}
\begin{document}

\title{Szenario 4. Schlechteste Angenommene Entwicklung}

\maketitle





\subsection{Bewertung}\label{H9174878}



\begin{tabu} to \textwidth { |X|X|X|X| }
\hline



Übertragung   & Immunevasion & Wesentliche Schwere & Erlebter Schwergrad
 \\


Rot & Rot & Rot & Rot
 \\
\hline

\end{tabu}




\subsection{Beschreibung}\label{H1664473}



Die hohe globale Inzidenz, die unvollständige globale Impfung und die Zirkulation in Tierreservoiren führen zum wiederholten Auftreten von Varianten, auch durch Rekombination (Austausch von genetischem Material zwischen verschiedenen Varianten, die dieselbe Zelle infizieren). 


Nicht alle Varianten sind gleich schwerwiegend, aber einige zeigen eine erhebliche Immunevasion in Bezug auf die  durch Impfstoffe  und frühere Infektionen erreichte Immunität. 


Unvorhersehbare Veränderungen in der Art und Weise, wie das Virus Krankheiten auslöst, verändern die Häufigkeit und das Altersprofil schwerer Erkrankungen und der Sterblichkeit, wobei die langfristigen Auswirkungen nach einer Infektion zunehmen. 


Die flächendeckenden jährlichen Impfung mit aktualisierten Impfstoffen sind erforderlich. 


Antivirale Resistenz ist weit verbreitet. 


Freiwillige Schutzmaßnahmen sind weitgehend nicht vorhanden und/oder eine Quelle gesellschaftlicher Konflikte. 


Erheblicher Einsatz von NPIs ist erforderlich, insbesondere wenn neue Varianten die Aktualisierung des Impfstoffs überholen (und/oder Testverfahren versagen).


\textbf{Für die nächsten 12-18 Monaten wird erwartet:} Die Entwicklung führt zu einer sehr großen Infektionswelle mit einer Zunahme schwerer Erkrankungen in weiten Teilen der Bevölkerung, auch wenn die schwersten gesundheitlichen Folgen weiterhin vor allem bei denjenigen zu beobachten sind, die keine vorherige Immunität besitzen.

\end{document}
