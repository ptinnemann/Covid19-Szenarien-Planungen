\documentclass{article}

\usepackage{tabu}
\begin{document}

\title{Szenario 3: Gefährliche Entwicklung}

\maketitle





\textbf{Bewertung}


\begin{tabu} to \textwidth { |X|X|X|X| }
\hline



Übertragung   & Immunevasion & Wesentliche Schwere & Erlebter Schwergrad
 \\


Rot & Rot & Orange & Orange
 \\
\hline

\end{tabu}




\textbf{Beschreibung}


Die hohe globale Inzidenz in Verbindung mit der zunehmenden Immunität der Bevölkerung führt über viele Jahre hinweg zu unvorhersehbaren Auftreten von Varianten mit einer Kombination aus verstärkter Immunabwehr und größerer Übertragbarkeit im Vergleich zu Omicron, manchmal mehr als einmal pro Jahr und/oder mit einem ähnlichen Schweregrad wie Delta in schlechten Jahren. 


Bestehende Immunität und aktualisierte Impfstoffe bieten weiterhin einen guten Schutz gegen die meisten schweren Verläufe. 


Obwohl nicht schwerwiegend, verursachen wiederholte Infektionswellen weit verbreitete Störungen mit unverhältnismäßig starken Auswirkungen auf bestimmte Gruppen, z. B. Kinder in der Ausbildung. 


Weitverbreitete jährliche Impfung mit aktualisierten Impfstoffen. 


Antivirale Resistenz ist weit verbreitet. 


SARS-CoV-2-Wellen führen nicht zu einer Verringerung der Influenza; SARS-CoV-2-Wellen überschneiden sich, was zu einer weiteren Belastung des Gesundheitswesens führt. 


Begrenztes freiwilliges Schutzverhalten während der Wellen. 


Einige Länder verhängen in schlechten Jahren strengere nicht-pharmazeutische Maßnahmen.


Die Erwartungen für die nächsten 12-18 Monaten sind: Das Auftreten  neuer bedenklicher Variante führt zu einer großen Infektionswelle, möglicherweise kurzfristig und außerhalb des Herbstes/Winters. 


Schwere Erkrankungen und die Sterblichkeit konzentrieren sich jedoch weiterhin auf bestimmte Gruppen (und sind niedriger als vor der Impfung), z. B. ungeimpfte, gefährdete und ältere Menschen.







\end{document}
