\documentclass{article}

\begin{document}

\title{Covid-19 Medium-Term Scenarios – February 2022}

\maketitle





(Reference: https://assets.publishing.service.gov.uk/government/uploads/system/uploads/attachment\_data/file/1054323/S1513\_Viral\_Evolution\_Scenarios.pdf)


This note sets out a range of scenarios to illustrate possible courses of the SARS-CoV-2 pandemic for the UK. All assume that SARS-CoV-2 will continue to circulate for the foreseeable future and that variants will emerge. These are scenarios that illustrate a range of possible futures but are not the only plausible courses that the pandemic could take. Shifts from one scenario to another over time are also possible. An outcome that lies outside the range covered by the four scenarios –better than the reasonable best-case scenario or worse than the reasonable worst-case scenario – cannot be ruled out. In each scenario, it is assumed that a relatively stable, repeating pattern is reached over time (2-10 years) but it is likely that the transition to this will be highly dynamic and unpredictable. It may not be possible to know with confidence from what happens in the next 12-18 months which long-term pattern will emerge. 


Viral evolution and immunity 


Infections are expected to occur in waves in all scenarios. The scenarios differ principally in the impact of individual waves (reflecting the transmissibility and severity of the associated variants) as well as their frequency and timing. Early waves (e.g. Alpha, Delta) were driven largely by the increased transmissibility of these variants. As global immunity increases from vaccination and infection, immune escape and waning immunity will become more important factors. Heterogeneity of global immunity is likely to continue, with varied protection against different variants around the world, potentially leading to extended co-circulation of more than one variant. There is also a feedback loop between the global epidemic dynamics (i.e. transmission and incidence) and viral evolution and adaptation: higher global SARS-CoV-2 prevalence provides more opportunities for viral evolution, while new variants can drive higher prevalence. Global vaccination rates will be key, and if variants of concern can originate in immunocompromised hosts, high rates of untreated HIV globally, for example, may be a risk factor. It is assumed, however, that the relationships among viral variant characteristics such as antigenic escape, transmissibility, severity and antiviral drug resistance are not necessarily predictive of each other. For example, higher transmissibility does not necessarily mean lower severity or vice versa. 


Interaction with other viruses 


The degree of seasonality of infection (and how quickly this emerges) is also important, as is the interaction of SARS-CoV-2 with other respiratory viruses such as influenza and RSV. Where influenza and SARS-CoV-2 waves co-occur, this would be expected to lead to a shorter, higher peak of total respiratory infections. If sequential, the health system will face a longer, drawn-out winter peak. Higher rates of infection from one virus could supress those from another leading to sequential infection patterns. Suppression of a virus in one year could then lead to loss of immunity in the population, increasing the risk of a severe wave the following year. 


Countermeasures 


Surveillance, vaccines, therapeutics and testing will also have large impacts on outcomes. Waves will be worse if detected late, vaccine effectiveness is low, or if stocks of effective vaccines are low or cannot be deployed quickly. Waves may be exacerbated in communities with lower vaccinations rates, which also tend to be the most disadvantaged. Lower vaccine effectiveness will also increase reliance on antiviral drugs, extensive use of which will increase the risk of resistance developing. Access to testing has also been key for reducing transmission and is likely to impact the shape and duration of any future waves. Overall, the speed at which testing, vaccination and antiviral provision can be ramped up in an emergency will significantly affect outcomes. No assumptions are made about what the ideal vaccination strategy will be (e.g. boosting using existing vaccines vs updated vaccines or entirely new vaccines) but it is likely that any significant new variant will spread and reach a peak of infections more quickly than a new vaccine can be produced, tested and distributed at scale. Protective behaviours are also expected to continue to play an important role in reducing transmission but cannot be reliably predicted for future waves.

\end{document}
