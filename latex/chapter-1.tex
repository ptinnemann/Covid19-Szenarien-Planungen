\documentclass{article}

\usepackage{hyperref}
\begin{document}

\title{Einführung: Covid-19 Szenarien}

\maketitle





Stand:  März 2022 


(Das Dokument basiert auf Vorarbeiten englischer Experten:innen die hier veröffentlicht sind: \href{https://assets.publishing.service.gov.uk/government/uploads/system/uploads/attachment_data/file/1054323/S1513_Viral_Evolution_Scenarios.pdf}{https://assets.publishing.service.gov.uk/government/uploads/system/uploads/attachment\_data/file/1054323/S1513\_Viral\_Evolution\_Scenarios.pdf}) 


\subsection{Hintergrund}\label{H1657566}



Es sollen einige Szenarien dargestellt werden, die den möglichen Verlauf der SARS-CoV-2-Pandemie für die Bundesrepublik Deutschland veranschaulichen. Alle Szenarien gehen davon aus, dass SARS-CoV-2 in absehbarer Zeit weiter zirkulieren wird und dass weitere neue Varianten auftreten werden. Die Szenarien veranschaulichen eine Reihe möglicher Entwicklungen, sind aber nicht die einzigen plausiblen und möglichen Entwicklungen, die die Pandemie nehmen könnte. Auch Übergänge von einem Szenario zum anderen sind im Laufe der Zeit möglich. 


Weitere Szenarien, das außerhalb der vier dargestellten Szenarien - also besser als das beste angenommen oder schlechter als das schlechteste Szenario - können nicht ausgeschlossen werden.


Bei jedem Szenario wird davon ausgegangen, dass im Laufe der Zeit (2-10 Jahre) ein relativ stabiles, sich wiederholendes Muster erreicht wird, aber es ist wahrscheinlich, dass der Übergang zu diesem Muster sehr dynamisch und unvorhersehbar sein wird. 


Aus dem, was in den nächsten 12-18 Monaten bis zum Herbst/Winter 2022 geschieht, lässt sich nicht mit Sicherheit ableiten, welches langfristige Muster in er SARS-CoV-2 Pandemie sich in Deutschland oder global herausbilden wird.


\subsection{Virale Entwicklung und Immunität}\label{H7213471}



In allen Szenarien wird erwartet, dass Infektionen wie bisher in Wellen auftreten. Die Szenarien unterscheiden sich vor allem in den Auswirkungen von der einzelnen Wellen (je nach Übertragbarkeit und Schweregrad der damit verbundenen Varianten) sowie in ihrer Häufigkeit und Zeitpunkt.


Frühe Wellen (z. B. Alpha, Delta) wurden weitgehend durch die erhöhte Übertragbarkeit dieser Varianten ausgelöst. In dem Maße, in dem die globale Immunität durch Impfung und Infektion zunimmt, werden Immunevasion und nachlassende Immunität zu wichtigeren Faktoren. Die Heterogenität der globalen Immunität in unterschiedlichen Ländern und Regionen der Welt wird sich wahrscheinlich fortsetzen, mit unterschiedlichem Schutz gegen verschiedenen Varianten weltweit, was zu einer ausgedehnten Ko-Zirkulation von mehr als einer Variante führen kann. 


Es wird erwartet, dass es auch eine Rückkopplungsschleife zwischen der globalen Epidemiedynamik (d. h. Übertragung und Inzidenz) und der viralen Evolution und Anpassung geben kann die zu einer höheren globale SARS-CoV-2-Prävalenz führt und  mehr Möglichkeiten für eine virale Evolution bietet, während neue Varianten die Prävalenz erhöhen können. Die Impfrate weltweit werden weiterhin eine wichtige Rolle spielen und die lokalen Effekte der Pandemie beeinflussen, und wenn bedenkliche Varianten (variants of concern, VoC) in immungeschwächten Wirten entstehen können, kann beispielsweise eine hohe Rate unbehandelter HIV-Infektionen weltweit ein Risikofaktor sein. Es wird jedoch davon ausgegangen, dass die Beziehungen zwischen den Merkmalen der viralen Varianten wie Antigen-Escape, Übertragbarkeit, Schweregrad und Resistenz gegen antivirale Medikamente nicht unbedingt voneinander abhängen. So bedeutet beispielsweise eine höhere Übertragbarkeit nicht unbedingt einen geringeren Schweregrad oder umgekehrt. 


\subsection{Interaktion mit anderen Viren}\label{H2543508}



Der Grad der Saisonalität der Infektion (und wie schnell diese auftritt) ist ebenfalls wichtig, ebenso wie die Interaktion von SARS-CoV-2 mit anderen Atemwegsviren wie Influenza und RSV. Treten Influenza- und SARS-CoV-2-Wellen gleichzeitig auf, dürfte dies zu einer kürzeren, höheren Spitze der gesamten Atemwegsinfektionen führen. Wenn sie nacheinander auftreten, wird das Gesundheitssystem mit einer längeren, langwierigen Winterspitze konfrontiert sein. 


Höhere Infektionsraten durch ein Virus könnten die Infektionen durch ein anderes verdrängen, was zu aufeinander folgenden Infektionsmustern führt. Die Unterdrückung eines Virus in einem Jahr könnte dann zu einem Verlust der Immunität in der Bevölkerung führen, was das Risiko einer schweren Welle im folgenden Jahr erhöht.


\subsection{Gegenmaßnahmen}\label{H976262}



Überwachung, Impfstoffe, Therapeutika und Tests werden ebenfalls einen großen Einfluss auf die Auswirkungen der Welleverläufe haben. Die Wellen werden schlimmer verlaufen, wenn sie zu spät erkannt werden, die Wirksamkeit der Impfstoffe gering ist oder wenn die Vorräte an wirksamen Impfstoffen gering sind oder nicht schnell bereitgestellt werden können. 


Die Wellen können sich in Bevölkerungsgruppen mit niedrigeren Impfraten, die in der Regel auch die am stärksten benachteiligten sind, verschlimmern. Eine geringere Wirksamkeit der Impfstoffe führt auch zu einer stärkeren Abhängigkeit von antiviralen Medikamenten, deren übermäßiger Einsatz das Risiko einer Resistenzentwicklung erhöht. 


Der Zugang zu Tests ist ebenfalls entscheidend für die Verringerung der Übertragungen und wird sich wahrscheinlich auf die Form und Dauer künftiger Wellen auswirken. Insgesamt wird die Geschwindigkeit, mit der Tests, Impfungen und die Bereitstellung von Virostatika in einem Notfall hochgefahren werden können, die Auswirkungen der Wellen beeinflussen. 


Es könne keine Annahmen darüber getroffen weren, wie die ideale Impfstrategie aussehen wird (z. B. Auffrischung mit vorhandenen Impfstoffen gegenüber aktualisierten Impfstoffen oder völlig neuen Impfstoffen), aber es ist wahrscheinlich, dass sich jede signifikante neue Variante schneller ausbreitet und einen Infektionsgipfel erreicht, als das ein neuer Impfstoff produziert, getestet und verteilt werden kann. 


Auch die nicht-pharmazeutischen Schutzmaßnahmen sowie deren einhalten dürfte weiterhin eine wichtige Rolle bei der Verringerung der Übertragung spielen, lässt sich aber für künftige Wellen nicht zuverlässig vorhersagen.

\end{document}
