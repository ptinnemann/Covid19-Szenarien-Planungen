\documentclass{article}

\usepackage{tabu}
\begin{document}

\title{Szenario 2: Zunehme Schwere Entwicklung}

\maketitle





\subsection{Bewertung}\label{H9902957}



\begin{tabu} to \textwidth { |X|X|X|X| }
\hline



Übertragung   & Immunevasion & Wesentliche Schwere & Erlebter Schwergrad
 \\


Orange & Orange & Orange & Grün
 \\
\hline

\end{tabu}




\subsection{Beschreibung}\label{H3906327}



Die zunehmende globale Immunität führt zu einem allgemein geringeren Schweregrad der Erkrankung. 


Infektionswellen werden durch Zyklen deutlich abnehmender Immunität und/oder das Auftreten neuer Varianten entweder von Omicron oder anderen Linien verursacht. 


Das allgemeine Muster ist eine jährliche saisonale Infektion mit guten und schlechten Jahren, wobei letztere eine hohe Übertragbarkeit und einen ähnlichen Schweregrad wie Delta aufweisen. 


Schwere Erkrankungen und Todesfälle beschränken sich weitgehend auf gefährdete, ältere Menschen und Personen ohne vorherige Immunität. 


Regelmäßig aufgefrischte Impfstoffe werden jährlich an gefährdete Personen und in schlechten Jahren an andere Personen verabreicht. 


Das freiwillige Schutzverhalten ist während der Wellen hoch. 


Einige Länder schreiben in schlechten Jahren nicht-pharmazeutische Maßnahme (z. B. Gesichtsbedeckung) vor. 


Antivirale Resistenzen treten auf und schränken die Anwendung ein, bis Kombinationstherapien verfügbar sind.


\textbf{Für die nächsten 12-18 Monaten wird erwartet:} Saisonale Infektionswelle im Herbst/Winter mit vergleichbarem Ausmaß und realisiertem Schweregrad wie bei der Omicron-Welle 2021/22.

\end{document}
